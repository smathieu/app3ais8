\documentclass[12pt,letterpaper]{article}
\usepackage{amsmath}
\usepackage{fancyhdr}
\usepackage{graphicx}
\usepackage{alltt}
\usepackage{color}
\usepackage{colortbl}
\usepackage{fullpage}
\usepackage{setspace}
\usepackage{pstricks}
\usepackage{verbatim}
\usepackage{comment}
\usepackage{listing}
\usepackage{framed}
\usepackage{listings}
\usepackage{longtable}
\usepackage{pdflscape}
\usepackage{multirow}
\usepackage[config=altsf]{subfig}
\usepackage[utf8]{inputenc}
\usepackage[francais]{babel}
\usepackage[plainpages=false,pdfpagelabels,hypertexnames=false]{hyperref}

%For pdf selection
\usepackage[T1]{fontenc}
\usepackage{lmodern}

%%%%% STYLE %%%%%%%
\topmargin	0in
\topskip	0in
\headheight	0in
\headsep	0in
\parindent	0in
\topsep		0in
\parskip	8pt
\floatsep	0in
%%%%%%%%%%%%%%%%%%%%

%%% SETUP HYPERLINK %%%%%
\hypersetup{
colorlinks 	= true,
linkcolor 	= black}
%%%%%%%%%%%%%%%%%%%%%%%%%

\begin{document}

%%%%%%% COMMANDS %%%%%%%%
\renewcommand{\labelitemi}{$\bullet$}
\newcommand{\unit}[1]{\ \mathrm{#1}}
\newcommand{\degree}{\ensuremath{^\circ}}
%%%%%%%%%%%%%%%%%%%%%%%%%

%%%%%%%%%%%%%%%%%%%%% PAGE TITRE %%%%%%%%%%%%%%%%%%%%%%%%%%%%%%%%%%%%
%%%%%%%%%%%%%%%%%%%%%%%%%%%%%%%%%%%%%%%%%%%%%%%%%%%%%%%%%%%%%%%%%%%%%
\thispagestyle{empty}
\begin{center}
	\vspace{20pt}
	\large{\textsc{
		Intelligence artificielle bio-inspirée\\
	}}
	\vspace{20pt}
	\large{\textsc{
		P02
	}}
	\vfill
	\begin{tabular}{ll}
      Simon Mathieu & 04 450 409 \\
      Steven Denis & 05 667 682 \\
	  Michael Janelle-Montcalm & 04 526 123 \\
	  Martin Provencher &	05 666 488 \\
	\end{tabular}
	\vfill
	Novembre 2009 \\
	\textbf{Université de Sherbrooke}
	\vspace{20pt}
\end{center}
\clearpage
%%%%%%%%%%%%%%%%%%%%% TABLE DES MATIÈRES %%%%%%%%%%%%%%%%%%%%%%%%%%%%
%%%%%%%%%%%%%%%%%%%%%%%%%%%%%%%%%%%%%%%%%%%%%%%%%%%%%%%%%%%%%%%%%%%%%
\begin{spacing}{0.1}
\tableofcontents
\end{spacing}
\clearpage

\section{Introduction}

\section{Analyse des données} % Steven
% Similitudes entre les canaux (redondance entre 1 et 6, on garde 6)
% Valeurs des maxima et minima sont plus grandes lors d'une chute que lors 
% d'une non-chute

\section{Hypothèses simplificatrices}

\subsection{Logique floue}

\subsection{Réseau de neurones} % Mike
% Utilisation seulement des max et min permet d'identifier les chutes

\section{Représentation de l'information}

\subsection{Logique floue}

\subsection{Réseau de neurones} % Steven
% Schéma-bloc du système
% Extraction des caractéristiques (max, min)
% Entrées et sorties (fichiers de données, sorties du script)

\section{Mise en oeuvre}

\subsection{Logique floue}

\subsection{Réseau de neurones} % Mike
% Données d'entraînement (sujet 5)
% Description de l'évolution du réseau (époques)
% Paramètres d'entraînement (momentum, learning rate)
% Ne converge pas toujours
% – Loi d’apprentissage, nombre d’unités cachées, nombre d’unités de sortie a expérimenter ;
% – Critères d’entraînement et d’évaluation ;
% – Création des ensembles d’entraînement et de test en lien avec l’apprentissage ;
% – Critère de classification et de reconnaissance.

\subsection{Algorithme génétique}

\section{Évaluation des performances}

\subsection{Logique floue}

\subsection{Réseau de neurones} % Mike
% Taux d'identification (avec explications des variations selon les paramètres)
% Performance avec les données d'entraînement, puis les données de test
% Différence des résultats selon le nombre d'unités cachées

\subsection{Algorithme génétique}

\section{Observations et perspectives futures}

\end{document}
